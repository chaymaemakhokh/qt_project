\documentclass[11pt,a4paper]{article}

\usepackage[utf8]{inputenc}
\usepackage[T1]{fontenc}
\usepackage[french]{babel}
\usepackage{geometry}
\geometry{margin=2.5cm}
\usepackage{hyperref}
\usepackage{array}
\usepackage{longtable}
\usepackage{graphicx}
\usepackage{amsmath}

\title{Documentation du modèle \\
Système de Gestion d'Entrepôt (Warehouse Manager)}
\author{Chaymae MAKHOKH \and Anas BZIOUI}
\date{\today}

\begin{document}
\maketitle
\tableofcontents
\newpage


Le modèle implémente les classes suivantes, telles que définies dans le diagramme UML~:

\begin{itemize}
  \item \texttt{ProduitAvecCaracteristiques}
  \item \texttt{ProduitAvecCycleDeVie}
  \item \texttt{Product}
  \item \texttt{ContrainteCompatibilite}
  \item \texttt{ReglesCompatibilite}
  \item \texttt{ElementsPalette}
  \item \texttt{Palette}
  \item \texttt{Conteneur}
  \item \texttt{Entrepot}
  \item les énumérations \texttt{TypeProduit}, \texttt{TypeConteneur}, \texttt{EtatProduit}
\end{itemize}

Toutes les classes du modèle sont regroupées dans le dossier
\verb|src/domain| du projet Qt.

\section{Vue d'ensemble du modèle}

Le modèle représente la structure suivante~:

\begin{itemize}
  \item Un \textbf{Entrepôt} (\texttt{Entrepot}) contient plusieurs
        \textbf{Conteneurs} (\texttt{Conteneur}) et plusieurs
        \textbf{Palettes} (\texttt{Palette}).
  \item Chaque \textbf{Conteneur} contient une liste de \textbf{Produits}
        (\texttt{Product}).
  \item Chaque \textbf{Palette} contient un ensemble d'\textbf{Éléments de palette}
        (\texttt{ElementsPalette}) qui eux-mêmes référencent des produits.
  \item Les \textbf{Produits} sont décrits par deux composantes :
        \texttt{ProduitAvecCaracteristiques} et \texttt{ProduitAvecCycleDeVie}.
  \item Les compatibilités entre produits sont définies par des
        \texttt{ContrainteCompatibilite}, regroupées dans
        \texttt{ReglesCompatibilite}.
\end{itemize}

\section{Énumérations}

\subsection{\texttt{TypeProduit}}

\paragraph{Rôle}
Enumération représentant le type d'un produit. Elle correspond au bloc
\emph{<<enum>> TypeProduit} du diagramme.

\paragraph{Valeurs}
\begin{itemize}
  \item \texttt{Alimentaire}
  \item \texttt{Electronique}
  \item \texttt{Medicament}
  \item \texttt{Autre} (valeur par défaut pour les cas non précisés)
\end{itemize}

\paragraph{Utilisation}
\begin{itemize}
  \item Attribut \texttt{type} de la classe \texttt{Product}.
  \item Attributs \texttt{typeA} et \texttt{typeB} de la classe
        \texttt{ContrainteCompatibilite}.
  \item Utilisé dans les méthodes de compatibilité de \texttt{ReglesCompatibilite}.
\end{itemize}

\subsection{\texttt{TypeConteneur}}

\paragraph{Rôle}
Enumération représentant le type d'un conteneur (normal, froid, fragile, etc.).
Elle correspond au bloc \emph{<<enum>> TypeConteneur} dans le diagramme.

\paragraph{Valeurs}
\begin{itemize}
  \item \texttt{Normal}
  \item \texttt{Froid}
  \item \texttt{Fragile}
  \item \texttt{Autre}
\end{itemize}

\paragraph{Utilisation}
Attribut \texttt{type} dans la classe \texttt{Conteneur}.

\subsection{\texttt{EtatProduit}}

\paragraph{Rôle}
Enumération représentant l'état d'un produit. Elle correspond au bloc
\emph{<<enum>> EtatProduit} du diagramme.

\paragraph{Valeurs}
\begin{itemize}
  \item \texttt{Stocke}
  \item \texttt{Expedie}
\end{itemize}

\paragraph{Utilisation}
\begin{itemize}
  \item Attribut \texttt{etat} dans \texttt{ProduitAvecCaracteristiques}.
  \item Attribut \texttt{etat} dans \texttt{ProduitAvecCycleDeVie}.
\end{itemize}

\section{Classes de données simples}

\subsection{\texttt{ProduitAvecCaracteristiques}}

\paragraph{Rôle}
Classe de données représentant les caractéristiques physiques et les conditions
de conservation d'un produit. Elle correspond à la classe
\emph{ProduitAvecCaracteristiques} reliée à \texttt{Product} par une
composition (losange plein) dans le diagramme UML.

\paragraph{Attributs}
\begin{itemize}
  \item \texttt{double m\_poids}~: poids du produit.
  \item \texttt{double m\_volume}~: volume occupé par le produit.
  \item \texttt{QString m\_conditionsConservation}~: texte décrivant les
        conditions de conservation (ex. \og -18°C \fg).
  \item \texttt{EtatProduit m\_etat}~: état du produit (stocké/expédié).
\end{itemize}

\paragraph{Méthodes principales}
La classe est implémentée uniquement dans un fichier d'en-tête
(\texttt{produit\_caracteristiques.h}) avec des accesseurs inline~:

\begin{itemize}
  \item Getters et setters pour chaque attribut~:
        \texttt{poids()}, \texttt{setPoids()}, \texttt{volume()},
        \texttt{setVolume()}, \texttt{conditionsConservation()},
        \texttt{setConditionsConservation()}, \texttt{etat()},
        \texttt{setEtat()}.
\end{itemize}

\paragraph{Lien avec le diagramme}
\begin{itemize}
  \item Tous les attributs de la boîte UML \emph{ProduitAvecCaracteristiques}
        sont présents.
  \item La composition 1--1 avec \texttt{Product} est réalisée par deux
        attributs privés dans \texttt{Product}~:
        \texttt{ProduitAvecCaracteristiques m\_caracteristiques;} et
        des méthodes d'accès \texttt{caracteristiques()}.
\end{itemize}

\subsection{\texttt{ProduitAvecCycleDeVie}}

\paragraph{Rôle}
Classe de données représentant le cycle de vie d'un produit
(dates et état). Elle correspond à la classe
\emph{ProduitAvecCycleDeVie} reliée à \texttt{Product} par une composition
dans le diagramme.

\paragraph{Attributs}
\begin{itemize}
  \item \texttt{QDate m\_dateEntreeStock}
  \item \texttt{QDate m\_datePeremption}
  \item \texttt{EtatProduit m\_etat}
\end{itemize}

\paragraph{Méthodes}
Getters / setters inline dans \texttt{produit\_cycledevie.h}~:
\texttt{dateEntreeStock()}, \texttt{setDateEntreeStock()},
\texttt{datePeremption()}, \texttt{setDatePeremption()},
\texttt{etat()}, \texttt{setEtat()}.

\paragraph{Lien UML}
\begin{itemize}
  \item Reprise exacte des attributs du diagramme.
  \item Composition 1--1 avec \texttt{Product} via l'attribut
        \texttt{ProduitAvecCycleDeVie m\_cycleDeVie;} dans \texttt{Product}.
\end{itemize}

\subsection{\texttt{ElementsPalette}}

\paragraph{Rôle}
Élément de la classe \texttt{Palette} correspondant à la boîte
\emph{ElementsPalette} du diagramme (quantité + liste de produits).

\paragraph{Attributs}
\begin{itemize}
  \item \texttt{int m\_quantite}~: quantité de produits.
  \item \texttt{QList<Product*> m\_produits}~: liste de pointeurs vers les
        produits concernés.
\end{itemize}

\paragraph{Méthodes}
Getters / setters simples pour la quantité et la liste de produits, définis
dans \texttt{elements\_palette.h}.

\paragraph{Lien UML}
L'association 1--* entre \texttt{Palette} et \texttt{ElementsPalette} est
implémentée par l'attribut~:
\begin{itemize}
  \item \texttt{QList<ElementsPalette> m\_elements;} dans \texttt{Palette}.
\end{itemize}

\section{Classes métier principales}

\subsection{\texttt{Product}}

\paragraph{Rôle}
Classe centrale représentant un produit manipulé dans l'entrepôt.
Elle hérite de \texttt{QObject} pour pouvoir émettre le signal
\texttt{productChanged()} et être reliée à l'interface Qt par signaux/slots.
Elle correspond à la classe \emph{Produit} du diagramme UML.

\paragraph{Attributs privés}
\begin{itemize}
  \item \texttt{QString m\_idProduit} (UML~: \texttt{IdProduit : String})
  \item \texttt{QString m\_nom} (UML~: \texttt{Nom : String})
  \item \texttt{TypeProduit m\_type} (UML~: \texttt{type : TypeProduit})
  \item \texttt{double m\_capaciteMax} (UML~: \texttt{capaciteMax : Double})
  \item \texttt{ProduitAvecCaracteristiques m\_caracteristiques}
  \item \texttt{ProduitAvecCycleDeVie m\_cycleDeVie}
\end{itemize}

\paragraph{Méthodes publiques}
\begin{itemize}
  \item Accesseurs pour les attributs d'identité et de type~:
        \texttt{idProduit()}, \texttt{setIdProduit()}, \texttt{nom()},
        \texttt{setNom()}, \texttt{type()}, \texttt{setType()},
        \texttt{capaciteMax()}, \texttt{setCapaciteMax()}.
  \item Accès aux sous-objets~:
        \texttt{caracteristiques()}, \texttt{cycleDeVie()}.
  \item Raccourcis vers les attributs des sous-objets~:
        \texttt{poids()}, \texttt{volume()}, \texttt{dateEntreeStock()},
        \texttt{datePeremption()}, \texttt{etat()}.
  \item Méthodes métier~:
        \begin{itemize}
          \item \texttt{bool estPerime(const QDate\& aujourdHui) const;}
          \item \texttt{int joursAvantPeremption(const QDate\& aujourdHui) const;}
        \end{itemize}
\end{itemize}

\paragraph{Signal}
\begin{itemize}
  \item \texttt{void productChanged();}~: émis lorsque l'une des propriétés
        du produit est modifiée.
\end{itemize}

\paragraph{Lien UML}
\begin{itemize}
  \item Tous les attributs et types du bloc \emph{Produit} sont repris.
  \item Les compositions vers \texttt{ProduitAvecCaracteristiques} et
        \texttt{ProduitAvecCycleDeVie} sont implémentées par des membres
        internes.
  \item L'association 0..* vers \texttt{Conteneur} et \texttt{Palette}
        n'est pas codée directement dans \texttt{Product} mais via les
        collections dans \texttt{Conteneur} et \texttt{Palette} (relation
        unidirectionnelle côté conteneur/palette).
\end{itemize}

\subsection{\texttt{ContrainteCompatibilite}}

\paragraph{Rôle}
Représente une règle élémentaire de compatibilité entre deux types de produit.
Elle correspond à la classe \emph{ContrainteCompatibilite} du diagramme UML.

\paragraph{Attributs}
\begin{itemize}
  \item \texttt{TypeProduit m\_typeA} (UML~: \texttt{typeA : TypeProduit})
  \item \texttt{TypeProduit m\_typeB} (UML~: \texttt{typeB : TypeProduit})
  \item \texttt{bool m\_compatible} (UML~: \texttt{compatible : bool})
\end{itemize}

\paragraph{Méthodes}
\begin{itemize}
  \item Getters / setters pour les trois attributs.
  \item \texttt{bool concerne(TypeProduit a, TypeProduit b) const;}~:
        teste si la contrainte s'applique au couple de types donné (ordre
        indifférent).
\end{itemize}

\paragraph{Lien UML}
La classe reprend exactement les attributs de la boîte UML correspondante
et est utilisée par \texttt{ReglesCompatibilite} avec une multiplicité 1..*.

\subsection{\texttt{ReglesCompatibilite}}

\paragraph{Rôle}
Contient l'ensemble des règles de compatibilité entre types de produits.
Elle correspond à \emph{ReglesCompatibilite} dans le diagramme, en
association 1..* avec \texttt{ContrainteCompatibilite}.

\paragraph{Attributs}
\begin{itemize}
  \item \texttt{QList<ContrainteCompatibilite> m\_contraintes;}
\end{itemize}

\paragraph{Méthodes}
\begin{itemize}
  \item \texttt{void ajouterContrainte(const ContrainteCompatibilite\& c);}~:
        ajoute une règle.
  \item \texttt{bool areCompatible(TypeProduit a, TypeProduit b) const;}~:
        renvoie \texttt{true} si les deux types sont compatibles selon les
        contraintes enregistrées.
  \item \texttt{const QList<ContrainteCompatibilite>\& contraintes() const;}~:
        accès en lecture à la liste des contraintes.
\end{itemize}

\paragraph{Signal}
\begin{itemize}
  \item \texttt{void rulesChanged();}~: émis lorsque les règles sont modifiées.
\end{itemize}

\paragraph{Lien UML}
\begin{itemize}
  \item La relation 1..* vers \texttt{ContrainteCompatibilite} est implémentée
        par la liste \texttt{m\_contraintes}.
  \item \texttt{ReglesCompatibilite} est utilisée par \texttt{Entrepot} et
        \texttt{Palette} pour vérifier la compatibilité des produits lors de
        la construction des palettes (algorithmes FIFO/FEFO).
\end{itemize}

\subsection{\texttt{Palette}}

\paragraph{Rôle}
Représente une palette d'expédition, contenant un ensemble d'éléments
(\texttt{ElementsPalette}) et donc de produits. Elle correspond à la classe
\emph{Palette} du diagramme UML.

\paragraph{Attributs}
\begin{itemize}
  \item \texttt{QString m\_idPalette} (UML~: \texttt{IdPalette : String})
  \item \texttt{QString m\_destination} (UML~: \texttt{Destination : String})
  \item \texttt{QDate m\_dateEnvoiPrevue} (UML~: \texttt{dateEnvoiPrevue : QDate})
  \item \texttt{double m\_capaciteMax} (UML~: \texttt{capaciteMax : Double})
  \item \texttt{QList<ElementsPalette> m\_elements}
  \item Les contraintes de compatibilité sont gérées via \texttt{ReglesCompatibilite},
        passée en paramètre aux méthodes d'ajout.
\end{itemize}

\paragraph{Méthodes}
\begin{itemize}
  \item Accesseurs classiques pour les attributs d'identité et de date.
  \item \texttt{const QList<ElementsPalette>\& elements() const;} et
        \texttt{QList<ElementsPalette>\& elementsRef();} pour accéder aux éléments.
  \item \texttt{double poidsTotal() const;}~: calcule le poids total de la palette
        en sommant les produits contenus dans \texttt{m\_elements}.
  \item \texttt{bool peutAjouter(Product *p, const ReglesCompatibilite *regles) const;}~:
        vérifie si un produit peut être ajouté en respectant la capacité max
        et les règles de compatibilité.
  \item \texttt{bool ajouterProduit(Product *p, const ReglesCompatibilite *regles);}~:
        ajoute le produit à la palette (en augmentant la quantité si nécessaire).
\end{itemize}

\paragraph{Signal}
\begin{itemize}
  \item \texttt{void paletteChanged();}~: émis lorsqu'on modifie le contenu
        ou les propriétés de la palette.
\end{itemize}

\paragraph{Lien UML}
\begin{itemize}
  \item Les attributs correspondent à ceux de la boîte \emph{Palette}.
  \item La relation 1..* vers \texttt{ElementsPalette} est implémentée par
        \texttt{QList<ElementsPalette> m\_elements;} avec une multiplicité 0..*.
  \item La relation vers \texttt{ContrainteCompatibilite} est représentée
        par l'utilisation de \texttt{ReglesCompatibilite} lors des ajouts de produits.
\end{itemize}

\subsection{\texttt{Conteneur}}

\paragraph{Rôle}
Représente un conteneur physique dans l'entrepôt. Il correspond à la classe
\emph{Conteneur} du diagramme UML.

\paragraph{Attributs}
\begin{itemize}
  \item \texttt{QString m\_idConteneur} (UML~: \texttt{IdConteneur : String})
  \item \texttt{TypeConteneur m\_type} (UML~: \texttt{type : TypeConteneur})
  \item \texttt{double m\_capaciteMax} (UML~: \texttt{capaciteMax : Double})
  \item \texttt{QList<Product*> m\_produits} (UML~: \texttt{Produits : List<Produit>})
\end{itemize}

\paragraph{Méthodes}
\begin{itemize}
  \item Accesseurs pour \texttt{idConteneur}, \texttt{type},
        \texttt{capaciteMax}, \texttt{produits()}.
  \item \texttt{double poidsTotal() const;}~: somme des poids des produits.
  \item \texttt{bool peutAjouter(Product *p) const;}~: vérifie la capacité max.
  \item \texttt{bool ajouterProduit(Product *p);}~: ajoute un produit si possible.
  \item \texttt{void retirerProduit(Product *p);}~: retire un produit.
\end{itemize}

\paragraph{Signaux}
\begin{itemize}
  \item \texttt{void conteneurChanged();}
  \item \texttt{void produitAjoute(Product *p);}
  \item \texttt{void produitRetire(Product *p);}
\end{itemize}

\paragraph{Lien UML}
La composition entre \texttt{Entrepot} et \texttt{Conteneur} (1..* conteneurs
par entrepôt) est implémentée par l'attribut
\texttt{QList<Conteneur*> m\_conteneurs;} dans \texttt{Entrepot}. La relation
0..* vers \texttt{Product} est implémentée par \texttt{m\_produits}.

\subsection{\texttt{Entrepot}}

\paragraph{Rôle}
Racine du modèle hiérarchique~: représente l'entrepôt global. Il correspond
à la classe \emph{Entrepot} du diagramme UML et constitue le niveau 1 de
l'arborescence.

\paragraph{Attributs}
\begin{itemize}
  \item \texttt{QString m\_idEntrepot} (UML~: \texttt{IdEntrepot : String})
  \item \texttt{QString m\_nom} (UML~: \texttt{Nom : String})
  \item \texttt{QString m\_adresse} (UML~: \texttt{Adresse : String})
  \item \texttt{double m\_surface} (UML~: \texttt{Surface : Double})
  \item \texttt{QList<Conteneur*> m\_conteneurs} (UML~: \texttt{allConteneurs})
  \item \texttt{QList<Palette*> m\_palettes} (UML~: \texttt{allPalette})
  \item \texttt{ReglesCompatibilite m\_regles} (ensemble des contraintes).
\end{itemize}

\paragraph{Méthodes}
\begin{itemize}
  \item Accesseurs pour l'identité et l'adresse de l'entrepôt.
  \item Accès aux listes~: \texttt{const QList<Conteneur*>\& conteneurs() const;},
        \texttt{const QList<Palette*>\& palettes() const;}.
  \item \texttt{QList<Product*> tousLesProduits() const;}~: construit la liste
        de tous les produits à partir des conteneurs.
  \item Gestion des conteneurs~:
        \texttt{Conteneur* creerConteneur();},
        \texttt{void supprimerConteneur(Conteneur *c);}.
  \item Gestion des produits~:
        \texttt{Product* creerProduitDans(Conteneur *c);},
        \texttt{void supprimerProduitDe(Conteneur *c, Product *p);}.
  \item Gestion des palettes~:
        \texttt{Palette* creerPalette();},
        \texttt{void supprimerPalette(Palette *p);}.
  \item Accès aux règles de compatibilité~:
        \texttt{ReglesCompatibilite* regles();}.
  \item Algorithmes de construction automatique des palettes~:
        \begin{itemize}
          \item \texttt{void genererPalettesFIFO(double capacitePalette);}\\
                tri des produits par date d'entrée (\emph{First In First Out}).
          \item \texttt{void genererPalettesFEFO(double capacitePalette);}\\
                tri des produits par date de péremption (\emph{First Expire First Out}).
        \end{itemize}
\end{itemize}

\paragraph{Signal}
\begin{itemize}
  \item \texttt{void entrepotChanged();}~: émis lorsqu'on modifie la structure
        (ajout/suppression de conteneurs, produits, palettes).
\end{itemize}

\paragraph{Lien UML}
\begin{itemize}
  \item Les attributs correspondent au bloc \emph{Entrepot} du diagramme.
  \item Les associations 0..* vers \texttt{Conteneur}, \texttt{Product} et
        \texttt{Palette} sont implémentées via les listes
        \texttt{m\_conteneurs}, \texttt{m\_palettes} et la méthode
        \texttt{tousLesProduits()}.
  \item La relation avec \texttt{ReglesCompatibilite} est implémentée par
        un membre interne \texttt{m\_regles}.
\end{itemize}


\end{document}
